% !TEX program = xelatex
\documentclass[t,aspectratio=54]{beamer}
\mode<presentation>
\usepackage{Yang_renmin}

\title{国家创新系统监测}
\subtitle{——启发、实践与未来——}
\author[Yang]{刘\ 洋 \\ \texttt{2017201684@ruc.edu.cn}}
\date[Labmeeting]{05/17/2022 吴老师组会}

\begin{document}
    \begin{frame}
        \titlepage
    \end{frame}

    \begin{frame}
        \frametitle{目录}
        \tableofcontents[pausesections]
    \end{frame}

    \section{我们为什么要监测?}
		\subsection{起源}
    \begin{frame}
			\frametitle{我们为什么要监测}
			\framesubtitle{起源}
    \end{frame}  

    \begin{frame}
        \frametitle{What Are Prime Numbers?}
        \begin{definition}
          A \alert{prime number} is a number that has exactly two divisors.
        \end{definition}
    \end{frame}

    \begin{frame}
        \frametitle{What Are Prime Numbers?}
        \begin{definition}
          A \alert{prime number} is a number that has exactly two divisors.
        \end{definition}
        \begin{example}
            \begin{itemize}
                \item 2 is prime (two divisors: 1 and 2).
                  \pause
                \item 3 is prime (two divisors: 1 and 3).
                  \pause
                \item 4 is not prime (\alert{three} divisors: 1, 2, and 4).
            \end{itemize}
        \end{example}
    \end{frame}

    \begin{frame}
        \frametitle{There Is No Largest Prime Number}
        \framesubtitle{The proof uses \textit{reductio ad absurdum}.}
        \begin{theorem}
          There is no largest prime number.
        \end{theorem}
        \begin{proof}
          \begin{enumerate}
          \item<1-> Suppose $p$ were the largest prime number.
          \item<2-> Let $q$ be the product of the first $p$ numbers.
          \item<3-> Then $q + 1$ is not divisible by any of them.
          \item<1-> But $q + 1$ is greater than $1$, thus divisible by some prime
            number not in the first $p$ numbers.\qedhere
          \end{enumerate}
      \end{proof}
        \only<4->{The proof used \textit{reductio ad absurdum}.}
    \end{frame}
    
    \begin{frame}
        \frametitle{What’s Still To Do?}
        \begin{itemize}
        \item Answered Questions
          \begin{itemize}
          \item How many primes are there?
          \end{itemize}
        \item Open Questions
          \begin{itemize}
          \item Is every even number the sum of two primes?
          \end{itemize}
        \end{itemize}
    \end{frame}

    \begin{frame}
        \frametitle{What’s Still To Do?}
        \begin{columns}
          \column{.5\textwidth}
            \begin{block}{Answered Questions}
              How many primes are there?
            \end{block}
          \column{.5\textwidth}
            \begin{block}{Open Questions}
                Is every even number the sum of two primes?
            \end{block}
        \end{columns}
    \end{frame}


  \begin{thebibliography}{10}
    \bibitem{Goldbach1742}[Goldbach, 1742]
      Christian Goldbach.
      \newblock A problem we should try to solve before the ISPN ’43 deadline,
      \newblock \emph{Letter to Leonhard Euler}, 1742.
    \end{thebibliography}
  and he can then add a citation:
  \begin{block}{Open Questions}
    Is every even number the sum of two primes?
    \cite{Goldbach1742}
  \end{block}

\end{document}
